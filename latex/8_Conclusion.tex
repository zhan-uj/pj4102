\chapter{總結}
本專題主要研究有限元素法的應用方式,探討了分析過程和應用方式,因為現實生活中不存在理想狀態下的剛體,每件物品、每個零件都是由柔性體所組合而成,隨著時間或外力的影響下會產生許多變量,而有限元素法則是以偏微分方程為基礎組成的分析法,主要用來對多種變量進行求解,以求出柔性體的受力狀況,有限元素法能對複雜模型分析的緣故,在近代被廣泛的應用在各種領域中,在建築或機械等地方都可以看見其身影。\\

四足機器人為一種模仿動物運動的機器人,可以幫助人類執行許多任務,而此設計中有著穩定性及負載的需求,所以步行裝置選用了四連桿為基礎架構,我們將四連桿機構帶入GeoGebra進行路徑分析,可以得到此機構的順逆向運動學,放入CoppeliaSim的擬真環境並運用了Python的作為控制程式,找到了最大的受力位置及角度。為了分析結果的準確性,我們帶入了兩種分析軟體進行比較,查看了分析結果並探討造成差異的原因,在之中發現了原先的ABS材料無法承受所預定的安全係數,因此在幾經尋找過後將全部零件材質換成了硬度較高的PLA,使四足機器人在日常做動時較不易損壞。\\

經過了上述的動作,我們已經驗證了5倍自身重量下步行機構依然可以正常做動,為了近一步增加其性能,我們將零件做生成式設計的步驟,透過軟體的迭代設計過程,對零件做輕量化處理且再次進行有限元素分析,卻保在四足機器人在減重後還保有一定強度。\\

有限元素法及生成式設計的誕生,讓設計者可以在設計開發中找出接近真實的受力情況,大幅的減少了來回修正模型的時間,也不受限於設計者的想像,可以產生許多結果供以選擇。\
\newpage


